\documentclass[11pt, aspectratio=169]{beamer}
\usefonttheme[onlymath]{serif}
\usepackage[utf8]{inputenc}

\usepackage{booktabs}
\usepackage{adjustbox}
% \usepackage{subcaption}
% \usepackage{helvet}
% \usepackage[default]{lato}

\usepackage{color, colortbl}
\definecolor{Gray}{gray}{0.9}
\newcolumntype{g}{>{\columncolor{Gray}}c}


\newcommand{\Skip}{\vspace{1em}}
\newcommand{\Squish}{\vspace{-1em}}



%% Lists
\newenvironment{wideitemize}{\itemize\addtolength{\itemsep}{10pt}}{\enditemize}
\newenvironment{wideenumerate}{\enumerate\addtolength{\itemsep}{10pt}}{\endenumerate}
\newenvironment{widedescription}{\description\addtolength{\itemsep}{10pt}}{\enddescription}

%% Colors
\definecolor{blue}{RGB}{0,114,178}
\definecolor{red}{RGB}{213,94,0}
\definecolor{yellow}{RGB}{240,228,66}
\definecolor{green}{RGB}{0,158,115}

\hypersetup{
  colorlinks=false,
  linkbordercolor = {white},
  linkcolor = {blue}
}

%% Beamer Specific Stuff
\setbeamercolor{frametitle}{fg=blue}
\setbeamercolor{title}{fg=black}
% \setbeamertemplate{footline}[frame number]
\setbeamertemplate{footline}{%
   \raisebox{5pt}{\makebox[\paperwidth]{\hfill\makebox[10pt]{\scriptsize\insertframenumber}}}}
\setbeamertemplate{navigation symbols}{} 
\setbeamertemplate{itemize items}{-}
\setbeamercolor{itemize item}{fg=blue}
\setbeamercolor{itemize subitem}{fg=blue}
\setbeamercolor{enumerate item}{fg=blue}
\setbeamercolor{enumerate subitem}{fg=blue}
\setbeamercolor{button}{bg=MyBackground,fg=blue,}

\setbeamertemplate{itemize item}{$\bullet$}
\setbeamertemplate{itemize subitem}{$\circ$}     % hollow circle for second level
\setbeamertemplate{itemize subsubitem}{$\ast$}   % asterisk for third level

\title[]{\textcolor{blue}{Markov Chain Monte Carlo}}

\author[CM]{Charlie Murry}
\institute{Boston College}

% \author[PGP]{}
% \institute[FRBNY]{\small{\begin{tabular}{c c c}
% Author A &&  Paul Goldsmith-Pinkham  \\
% Somewhere Fancy && FRBNY \\ \\

% Author C && Author D   \\
% \multicolumn{3}{c}{Somewhere Fancy} \\
% \end{tabular}}}

\date{\today}


\begin{document}

%%% TIKZ STUFF
% \tikzset{   
%         every picture/.style={remember picture,baseline},
%         every node/.style={anchor=base,align=center,outer sep=1.5pt},
%         every path/.style={thick},
%         }
% \newcommand\marktopleft[1]{%
%     \tikz[overlay,remember picture] 
%         \node (marker-#1-a) at (-.3em,.3em) {};%
% }
% \newcommand\markbottomright[2]{%
%     \tikz[overlay,remember picture] 
%         \node (marker-#1-b) at (0em,0em) {};%
% }
% \tikzstyle{every picture}+=[remember picture] 
% \tikzstyle{mybox} =[draw=black, very thick, rectangle, inner sep=10pt, inner ysep=20pt]
% \tikzstyle{fancytitle} =[draw=black,fill=red, text=white]
%%%% END TIKZ STUFF

% Title Slide (will go at the bottom of title slide)
\begin{frame}
\maketitle
  % \centering The views expressed do not necessarily reflect the position of the Federal Reserve Bank of New York or the Federal Reserve System.
\end{frame}

% INTRO
% \section[intro]{Introduction} % (fold)
% \label{sec:introduction}

\begin{frame}[c]\frametitle{Overview of Bayesian Estimation}
    
\begin{itemize}
    \item We wish to know about unkown parameter $\theta^0\in R^N$
    \item We have data $y$, and $L(y|\theta)$ is the likelihood of $y$ given that $\theta = \theta^0$. 
\end{itemize}    

\textbf{Frequentist}
\begin{itemize}
	\item Derive an estimator (MLE) and analyze statistical properties of that estimator
	$$ \hat{\theta} = \max_{\theta} L(y | \theta).$$
\end{itemize}

\textbf{Bayesian}
\begin{itemize}
	\item Start with a prior belief, $p(\theta)$.
	\item Use data to update their belief  to posterior using Bayes Rule
	$$\pi(\theta | y) = \frac{L(y | \theta) p(\theta)}{f(y)}$$ 
where $f(y) = \int L(y|\theta)\pi(\theta) d\theta$ is the marginal distribution of $y$. 

\end{itemize}

\end{frame}


\begin{frame}[c]\frametitle{Bayes Rule}
    
$$\pi(\theta | y) = \frac{L(y | \theta) p(\theta)}{f(y)}$$ 

is just

$$Pr(A | B) = \frac{Pr(A \cap B)}{Pr(B)} = \frac{Pr(B|A) Pr(A)}{Pr(B)}.$$

\end{frame}


\begin{frame}[c]\frametitle{Bayesian Estimation Overview}
    
\begin{wideitemize}
    \item Outcome of estimation is $\rho(\theta)$: summarizes everything we know about where $\theta$ is.
    \item Typically report moments of $\rho$.
    \item $\rho$ is typically not tractable. \textbf{How do we report moments from something that is not tractable?} (it's not necessarily normally distributed with mean $\bar{\theta}$)
\end{wideitemize}    

\vspace{1em}
\textbf{Good:} No need to solve complex optimization problem.

\vspace{1em}
\textbf{Bad:} By construction there is a complex integral.

\end{frame}


\begin{frame}[t]\frametitle{Tractable Example I}
\framesubtitle{Cameron and Trivedi, p422}    

Nothing about a posterior per se that requires MCMC.


\begin{wideitemize}
	\item Suppose you observe $N$ draws from 
a normal distribution with mean $\theta$ and variance $\sigma^2$, e.g., 
$$y_i \sim N(\theta, \sigma^2).$$ 
	\item $\sigma^2$ is known, but you want to estimate $\theta$.
	\item A frequentist might use maximum likelihood estimation. The likelihood is:
\begin{align*}
L(y|\theta) &= \prod_{i=1}^N (2\pi \sigma^2)^{- \frac{1}{2} } \exp \left\{ - \frac{(y_i - \theta)^2}{2 \sigma^2} \right\} \\
&= (2\pi \sigma^2)^{- \frac{N}{2} } \exp  \left\{ - \sum_{i=1}^N \frac{(y_i - \theta)^2}{2 \sigma^2} \right\} \\
% & \propto  \exp  \left\{ \frac{-1}{2 \sigma^2}  \sum_{i=1}^N (y_i - \bar{y} +\bar{y} - \theta)^2 \right\} \\
% &  \propto  \exp  \left\{ - \frac{1}{2 \sigma^2}  \sum_{i=1}^N (y_i - \bar{y})^2 \right\} \exp  \left\{ - \frac{N}{2 \sigma^2} (\bar{y} - \theta)^2 \right\} \\
&  \propto  \exp  \left\{ - \frac{N}{2 \sigma^2}   (\bar{y} - \theta)^2 \right\} \\
\end{align*}
\end{wideitemize}


\end{frame}


\begin{frame}[t]\frametitle{Tractable Example II}
\framesubtitle{Cameron and Trivedi, p422}    

\begin{wideitemize}
	\item Clearly this is maximized at $\bar{y}$, which is the MLE estimate.
	\item Just compute the average from your data. 
\end{wideitemize}

\vspace{1em}
\textbf{Bayesian}
\begin{wideitemize}
	\item Define prior belief, $\theta$.
	\item Suppose that belief is normally distributed with mean $\mu$ and variance $\tau^2$
	\item Prior density:
	$$p(\theta) =  (2\pi \tau^2)^{- \frac{1}{2} } \exp \left\{ - \frac{(\theta - \mu)^2}{2 \tau^2} \right\}.$$

\end{wideitemize}

\end{frame}


\begin{frame}[t]\frametitle{Tractable Example III}
\framesubtitle{Cameron and Trivedi, p422}    

Following Bayes Rule, the Posterior is proportional to:

\begin{align*}
\pi(\theta | y) & \propto L(y|\theta) p(\theta) \\
& \propto \exp  \left\{ - \frac{N}{2 \sigma^2}   (\bar{y} - \theta)^2 \right\} \exp \left\{ - \frac{(\theta - \mu)^2}{2 \tau^2} \right\} \\
% & \propto \exp \left\{ - \frac{1}{2} \left[ \frac{  (\bar{y} - \theta)^2 }{N^{-1} \sigma^2} + \frac{(\theta - \mu)^2}{ \tau^2 } \right] \right\} \\
& \propto  \exp \left\{ - \frac{1}{2} \left[ \frac{ (\theta - \tilde{\mu})^2 }{\tilde{\tau}^2} \right] \right\}
\end{align*} 
Where,  
$$\tilde{\mu} = \tilde{\tau}^2 \left( \frac{N}{\sigma^2} \bar{y} + \frac{1}{\tau^2} \mu \right)$$
$$\tilde{\tau}^2 = \left(  \frac{N}{\sigma^2} + \frac{1}{\tau^2} \right)^{-1} $$


\end{frame}


\begin{frame}[c]\frametitle{Tractable Example III}
\framesubtitle{Cameron and Trivedi, p422}    

\begin{wideitemize}
	\item The final line is a normal kernel -- just complete the square :). 
	\item The posterior is normally distributed with mean $\tilde{\mu}$ which is a weighted sum of $y$ and the prior mean $\mu$.
	\item Since this posterior is normal, it is easy for us to compute the moments.
	\item Mean of posterior goes to $\bar{y}$ as $N\rightarrow\infty$. 
	\item But computing moments of even slightly messier posteriors will require complex integration that we will tackle via simulation. 
\end{wideitemize}

\end{frame}



\begin{frame}[c]\frametitle{Conjugate Priors}
  \framesubtitle{A Convenient Mathematical Property}
  
  Bayesian updating combines:
  $$ \text{posterior} \propto \text{likelihood} \times \text{prior} $$
  
  A conjugate prior occurs when:
  \begin{itemize}
  \item The posterior distribution is in the same family as the prior
  \item This makes computation of the posterior analytically tractable
  \end{itemize}
  
  \vspace{0.5em}
  Classic Example:
  $$ X_i \sim N(\mu, \sigma^2) \quad \text{(likelihood)} $$
  $$ \mu \sim N(\mu_0, \tau^2) \quad \text{(prior)} $$
  $$ \mu|X_1,\ldots,X_n \sim N(\mu_n, \tau_n^2) \quad \text{(posterior)} $$
  
  \vspace{0.5em}
  Key Benefits:
  \begin{itemize}
  \item Closed-form posterior updates
  \item No need for numerical integration
  \item Particularly useful in iterative procedures (e.g., Gibbs sampling)
  \end{itemize}
  \end{frame}





\begin{frame}[c]\frametitle{Review of Monte Carlo Integration}
    
The point of monte carlo integration is to use draws from a distribution to calculate the moments of $\rho(\theta|y)$. If $\rho(\cdot)$ is ``easy'' to draw from (say, uniform or normal) then we can use traditional monte carlo integration techniques: 
$$E[m(\theta)] = \int_\Theta m(\theta) \rho(\theta|y) d\theta \approx \frac{1}{S} \sum_{s=1}^S m(\theta_s)$$
\begin{itemize}
\item $m(\cdot)$ is an arbitrary function: e.g. $m(\theta)=\theta$ if we  want to identify the mean.  
\item $\theta_s$ is a draw from $\rho(\theta|y)$. 
\end{itemize}

However, this isn't helpful if we don't know how to generate draws from $\rho(\cdot)$ and if we did, we could probably just integrate it directly. 


\end{frame}



\begin{frame}[c]\frametitle{Markov Chain Monte Carlo}
 
MCMC uses draws from a \alert{Markov Chain}, instead of i.i.d. draws from some known distribution. 

\vspace{.5em} 
Use MCMC when:
\begin{itemize}
\item Analytic solutions aren't tractable. 
\item IID sampling doesn't give adequate coverage (perhaps dimension is too high or good approximation of $\rho$ is unknown). 
\end{itemize}


The goal becomes constructing an \alert{ergodic} Markov Chain $F$ (so that the \alert{stationary distribution} exists) such that the stationary
distribution is exactly $\rho$.  If we do this then we can generate moments of $\rho$ from

$$ E[m(\theta)] \approx \frac{1}{S} \sum_{i=1}^S m(\theta_i) $$
where $\theta_i \sim F(\cdot | \theta_{i-1})$. 

\end{frame}


\begin{frame}[c]\frametitle{English, please?}
    

\textbf{ergodic:} statistical properties can be deduced from a single, sufficiently long, random sample of the process. 

\vspace{2em}
\textbf{not ergodic:} a process that changes erratically at an inconsistent rate

\vspace{2em}
\textbf{stationary distribution:} probability distribution that remains unchanged in the Markov chain as time progresses. (The transition matrix of a discrete processes remains constant)

\end{frame}


\begin{frame}[c]\frametitle{Markov Chain Theory}    

Let the state space for $\theta$ be discrete, $\Theta = \{\theta^{(1)}, \ldots, \theta^{(K)} \}$.\footnote{You'll typically have a continuous distribution, but using a discrete example is much easier on the math (no measure theory).} 

\vspace{2em}
Let our chain be defined by
$$P(\theta_{r+1} = \theta^{(j)} | \theta_r = \theta^{(i)}) = p_{ij} $$
So the Markov transition matrix is, 
$$P = \begin{bmatrix} p_{11} & p_{12} & \ldots & p_{1K} \\
 p_{21} & p_{22} & \ldots&  p_{2K} \\
 \vdots & & & \vdots \\
 p_{K1} & p_{K2} & \ldots&  p_{KK} \end{bmatrix}$$

\end{frame}



\begin{frame}[c]\frametitle{MC Theory: Stationarity}
    

Let $\pi_0$ be an initial distribution over states (a $1 \times K$ vector). Then the distribution over states after 1 period will be:
$$Pr( \theta_1 = \theta^{(j)}) = \sum_{i=1}^K Pr(\theta_0 = \theta^{(i)}) p_{ij} = \sum_{i=1}^K \pi_{0i} p_{ij} $$
Or in matrix notation for the entire distribution, 
$$\pi_1 = \pi_0 P$$

\vspace{.5em}
If for all $i,j$: $p_{ij} > 0$, then every state will be visited infinitely often.

\vspace{1em}
A stationary distribution exists and is unique:
$$ \lim_{r \rightarrow \infty} \pi_0P^r = \pi$$
for any $\pi_0$. Then we will have 
$$ \pi = \pi P $$
The stationary distribution $\pi$ is sometimes called the ``invariant'' distribution. 

\end{frame}


\begin{frame}[c]\frametitle{Time Reversibility}
    
\begin{definition} A chain is time reversible with respect to $\pi$ if it has the same behavior backwards and forwards starting from $\pi$. That is if the chance of seeing a transition from $i$ to $j$ is the same as seeing a transition from $j$ to $i$: 
$$ \pi_i p_{ij} = \pi_j p_{ji}$$
\end{definition}


\end{frame}



\begin{frame}[c]\frametitle{Markov Chain Monte Carlo: Gibbs Sampling}

  This is a technique to simplify the Monte Carlo draws, by alternating draws from simpler conditionals. 

  Construct Markov chain by ``cycling'' through conditional distributions related to $\pi$. 

\begin{wideitemize}
	\item Let $\theta = [\theta_1, \theta_2]'$ with posterior density $p(\theta_1, \theta_2)$.
	\item \emph{If the conditional densities are known}, then alternating sequential draws from $p(\theta_1 \mid \theta_2)$ and $p(\theta_2 \mid \theta_1)$ converge to $p(\theta_1, \theta_2)$.
\end{wideitemize}


\end{frame}




\begin{frame}[c]\frametitle{Gibbs Example: Probit}
\framesubtitle{Using Data Augmentation}
    
Model: 
$$ z_i = x_i \beta + \epsilon_i $$
$$ y_i = \begin{cases} 0 & z_i \leq 0 \\ 1 & z_i > 0 \end{cases}$$
$$ \epsilon_i \sim N(0, 1)$$
We observe a random sample of $(y_i, x_i)$ and want to estimate $\beta$. \\

\vspace{1em}
Suppose we have a prior $\beta \sim N(\bar{\beta}, A^{-1})$. If we observed $z_i$ then the posterior would be normal (normal is the \emph{conjugate prior} of normal). 

\vspace{1em}
However, when $z$ is unobserved there is no simple conjugate prior.  

\vspace{1em}
Instead, we can use an ``augmentation step'' by employing a Gibbs sampler with two blocks $(z_i, \beta)$, the second step uses draws of $z$ and the normal conjugate prior.

\end{frame}


\begin{frame}[c]\frametitle{Probit Example | Algorithm}
    
\begin{enumerate}
\item Given $\beta_{r-1}$, draw $z_i$ by drawing from a truncated normal: 
$$ z_{i, r} | \beta_{r-1}, y_i, x_i \sim \mbox{TruncatedNormal}_a^b(-x_i \beta_{r-1}, 1) $$
Where bounds are $a = 0, b = \infty$ if $y_i = 1$ and $a = -\infty, b = 0$ if $y_i = 0$
\item Draw $\beta_r | z_{i,r}, x_i$ from the posterior of a regression of $z$ on $x$:
$$\beta_r \sim N(\tilde{\beta}, (X'X + A)^{-1})$$
where $\tilde{\beta} = (X'X + A)^{-1}(X'z + A\bar{\beta})$. 
\item After many draws, we have a sample of $\beta_r$ which we use as draws from the stationary distribution. 
\end{enumerate}    

\end{frame}


\begin{frame}[c]\frametitle{Another Example | Multivariate Normal}
    
The file {\tt simpleGibbs.m} implements Gibbs sampling to draw from a bivariate normal: 

$$(y_1, y_2)' \sim N\left( \begin{pmatrix} \mu_1 \\ \mu_2 \end{pmatrix}, \begin{pmatrix} \sigma_1^2 & \rho \sigma_1 \sigma_2 \\ \rho \sigma_1 \sigma_2 & \sigma_2^2 \end{pmatrix} \right) $$

This trivially implies conditional distributions: 
$$y_1 | y_2 \sim N(\mu_1 + \rho \frac{\sigma_1}{\sigma_2}(y_2 - \mu_2), \sigma_1^2(1 - \rho^2))$$

% Let's go through the code and see how it works: 
% Items to consider: 
% \begin{itemize}
% \item See what happens when you start from (50, 50). 
% \item See what happens when you raise $\rho$ to .95. 
% \item Choosing a burn-in period. 
% \item The relationship between correlation and convergence. 
% \item Checking autocorrelation of the chain: 
% $$s_{\theta_i}(k) = \frac{ \sum_{r = k+1}^R (\theta_r - \bar{\theta})(\theta_{r-k} - \bar{\theta})}{ \sum_r=1^R  (\theta_r - \bar{\theta})^2} $$
% \item Can also check for convergence by computing moments from different subsamples of the chain and making sure they are robust.
% \end{itemize}    


\end{frame}


\begin{frame}[c]\frametitle{Has My MCMC Chain Converged?}
    
Properties of MCMC rely on limit arguments.  

\Skip
The more complicated the chain, the harder it is to make sure you have ``converged.''    


\Skip
Some rules of thumb:
\begin{itemize}
  \item Use a ``burn-in'' period. 
  \item Plot time series of draws to make sure there is no trend. 
  \item Run chain from several start points. 
  \item Compare distributions from different subsamples of the chain. 
  \item Compute the autocorellation function: 
$$s_{\theta_i}(k) = \frac{ \sum_{r = k+1}^R (\theta_r - \bar{\theta})(\theta_{r-k} - \bar{\theta})}{ \sum_{r=1}^R  (\theta_r - \bar{\theta})^2} $$
To make sure correlation is dying as time between draws increases. 
\end{itemize}    


\end{frame}


\begin{frame}[c]\frametitle{Has My MCMC Chain Converged? Maybe. }
    
Sadly, there is no proof of convergence in an empirical application. 

\Skip
Similar to finding a global optimizer in frequentist approaches. 


\Skip
MCMC/Bayesian is not a free lunch. 

\end{frame}


\begin{frame}[c]\frametitle{Metropolis Algorithm I}
    
\begin{itemize}
  \item Gibbs sampling relies on being abble to draw from \emph{conditional} distributions.
  \item Idea from importance sampling (which we did not go over...)
  \begin{itemize}
     \item Draw from a known distribution and re-weight. 
   \end{itemize} 
\end{itemize}

\Skip
The Metropolis Algorithm constructs a sequence $\{\theta^{(n)}, n=1,2,... \}$ whose distributions converge to the target posterior. 


\end{frame}



\begin{frame}[c]\frametitle{Metropolis Algorithm II}
    
\begin{enumerate}
  \item Draw starting point $\theta^{(0)}$ from an initial approximation to the posterior for which  $p(\theta^{(0)})>0$ 
  \begin{itemize}
    \item Ex.: multivariate t-dist centered on mode of marginal posterior distribution.
  \end{itemize}
  \item Set $n=1$. Draw $\theta*$ from a \emph{symmetric} \textbf{jumping distribution}, $J_1(\theta^{(1)} \mid \theta^{(0)})$. 
  \begin{itemize}
    \item Ex: $\theta^{(1)} \mid \theta^{(0)}\sim \mathcal{N}(\theta^{(0)}, V)$ for a fixed V. 
  \end{itemize}
  \item Calculate ratio of densities: $r = p(\theta^*)/p(\theta^{(0)})$.
  \item Set $$\theta^{(1)} = \begin{cases} \theta^* \text{with prob.}\,\, \min(r,1) \\
          \theta^{(0)} \text{with prob.}\,\,\ (1-\min(r,1)) \end{cases}$$
          so the draw $\theta^{(1)}$ is a draw from a mixture distribution. 
  \item Return to step 2.
  \item Stop after many iterations. 

\end{enumerate}

\end{frame}


\begin{frame}[c]\frametitle{Metropolis Algorithm III}
    
This is \emph{just} a way to increase $p(\theta)$ iteratively. 

\Skip
\begin{enumerate}
  \item Start with a guess. 
  \item Update guess if 
  \begin{enumerate}
    \item always if likelihood ($p()$) increases or  
    \item with some probability if likelihood decreases. 
  \end{enumerate}
\end{enumerate}

\Skip
\textbf{Metropolis-Hastings}

\Skip
A refinement that uses a specific \textbf{jumping distribution} that is non-symmetric and therefore leads to more frequent transitions. 


\end{frame}



% \begin{frame}[c]\frametitle{Metropolis-Hastings Algorithm}
    
% \begin{itemize}
%   \item Gibbs sampling relies on being able to draw from \emph{conditional} distributions.
%   \item Idea from importance sampling (which we did not go over...)
%   \begin{itemize}
%      \item Draw from a known distribution and re-weight. 
%    \end{itemize} 
% \end{itemize}

% \Skip
% For simplicity, let's again do everything with discrete states. 

% \Skip
% Suppose we want to draw from $\pi$. But we only know how to draw from some candidate distribution $q_i$ (a known density that is potentially conditional on current state). 
% That is: 
% $$q_{ij} = Pr( \theta^{(j)} | \theta^{(i)})$$ 

% \end{frame}


% \begin{frame}[c]\frametitle{Metropolis-Hastings Algorithm}

% The MH algorithm generates a Time-Reversible Markov chain with respect to $\pi$ by following: 
% \begin{enumerate}
% \item Initialize $\theta_0$ to some initial state. 
% \item For $t = 1:T$:
% \begin{enumerate}
% \item Let $i$ be such that $\theta_{t-1} = \theta^{(i)}$, draw a candidate state $\tilde{\theta}$ from $q_i$,  
% \item Let $j$ be such that $\tilde{\theta} = \theta^{(j)}$, compute: $$\alpha_{ij} = \min\left\{ 1, \frac{\pi_j q_{ji}}{\pi_i q_{ij}} \right\}$$
% \item Draw: 
% $$\theta_t = \begin{cases}  \tilde{\theta} & \mbox{w.p. } \alpha_{ij} \\  \theta_{t-1} & \mbox{w.p. }  1 - \alpha_{ij} \end{cases} $$
% \end{enumerate}
% \end{enumerate}

% \end{frame}


\begin{frame}[c]\frametitle{Go over example}
    
[matlab code on website]

\end{frame}


\begin{frame}[c]\frametitle{Relationship to ``Simulated Annealing''}
\framesubtitle{and other stochastic optimization routines}    

A non-gradient iterative optimization routine. 

\begin{enumerate}
  \item Perturb one element of parameter vector.
  \item Replace if 
  \begin{itemize}
     \item The objective function improves or
     \item The objective function worsens but ``not by too much'' according to a ``Metropolis'' criteria
     $$ exp((Q_N(\theta^*_s) - Q_N(\hat{\theta}^*_s))/T_s) > u$$
     where $s$ is the iterative time, $u$ is a unit uniform draw, and $T_s$ the \emph{temperature}.
   \end{itemize} 
\end{enumerate}

\Skip
Downhill moves are accepted with a probability that decreases over time. 

\Skip
User chooses step size that governs perturbation and the temperature function. 

\Skip
Also see Chernozhukov and Hong, ``An MCMC approach to classical estimation'', \emph{J. of Econometrics}, 2003. 

\end{frame}




\end{document}

